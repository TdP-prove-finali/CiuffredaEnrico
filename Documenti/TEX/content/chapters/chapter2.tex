\chapter{Descrizione dettagliata del problema}
\label{sec:descrizione dettagliata del problema}


% use [] to set name for ToC
\section[Campionato di F1]{Campionato di F1} % ok with fontsize=12pt
Il campionato di F1 è un campionato mondiale automobilistico organizzato dalla FIA. Il campionato è composto da un numero variabile di gare che possono essere disputate nei vari Paesi del Mondo. In genere una settimana in cui vi è la gara è composta da 3 parti: 
\begin{itemize}
\item \textbf{Prove libere} (generalmente svolte il venerdì), dove vengono provati i vari setup per le macchine;
\item \textbf{Qualifiche} (generalmente svolte il sabato), dove i piloti si qualificano alla gara e cercano di avere il miglior posto possibile in griglia di partenza per la gara;
\item \textbf{Gara} (generalmente svolta la domenica).
\end{itemize}
Al termine del campionato vengono assegnati due tipi diversi di titoli: il \textbf{Premio Costruttore}, assegnato alla scuderia che ha totalizzato più punti nel corso del campionato e il  \textbf{Premio pilota}, assegnato al pilota che ha totalizzato più punti nel corso del campionato  .\\
A seguito della pandemia da Covid-19, il campionato di F1 2020 è stato caratterizzato da un numero ridotto di gare, dall'assenza di alcuni piloti in determinate gare e dall'introduzione di un sistema che ha come obiettivo quello di minimizzare il gap tra le varie scuderie limitando gli investimenti delle scuderie e limitando il numero di possibili sostituzioni nei pezzi che compongono la vettura.

\section[Obiettivi]{Obiettivi} % ok with fontsize=12pt
L'applicazione ha vari obiettivi che però possono essere raggruppati in due insiemi. Da una parte abbiamo gli obiettivi finalizzati ad aiutare una scuderia nelle scelte effettuate ad inizio anno, come possono essere quelle della scelta del pilota o della scelta della quantità di investimenti da effettuare. Inoltre poichè l'applicazione simula e riporta la classifica di  ogni gara, una scuderia può anche cercare di migliorare le prestazioni in quella determinata gara. D'altra parte abbiamo invece, gli obiettivi finalizzati ad un calcolo probabilistico della vittoria del titolo piloti, del titolo costrutturi e della griglia di partenza e della classifica finale di ogni gara; questi risultati, ottenuti dalla simulazione ripetuta più volte, possono essere ad esempio una base di partenza per calcolare la quotazione di ogni evento da parte dei bookmakers.
\section[Criticità]{Criticità} % ok with fontsize=12pt
L'applicazione presenta varie criticità dovute sopratutto all'impossibilità di recuperare qualsiasi tipo di dato relativo alle varie vetture poichè questi dati rimangono segreti all'interno delle varie scuderie e non vengono pubblicati. Non è quindi possibile analizzare le varie componenti e conseguentemente non è possibile nemmeno quantificare le prestazioni come ad esempio la velocità massima o la velocità in curva.\\
Un ulteriore criticità è rappresentata dall'irrealistica, ma efficace a livello simulativo, semplificazione dell'abilità di un pilota. Infatti l'applicazione non tiene conto delle varie qualità che un pilota può avere, un esempio potrebbe essere il caso in cui un pilota potrebbe schivare un altro pilota e quindi non effettuare un incidente grazie alla sua destrezza mentre nella stessa situazione un altro pilota potrebbe essere coinvolto poichè i suoi riflessi non sono così eccellenti.\\
Un ulteriore criticità è quella di non implementare una strategia di scuderia per ogni gara ma utilizzare una strategia fornita dai dati degli anni passati.\\
Le ultime due criticità sono rappresentate dal consumo degli pneumatici e dalla temperatura e dallo stato dell'asfalto.

