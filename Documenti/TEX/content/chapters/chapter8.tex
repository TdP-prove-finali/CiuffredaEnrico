\chapter{Conclusione e possibili miglioramenti}
\label{sec:Conclusione e possibili miglioramenti}

L'applicazione raggiunge tutti gli obiettivi che sono stati previsti in fase di progetto; tuttavia è possibile apportare vari miglioramenti ad essa. Alcuni esempi potrebbero essere l'introduzione di determinati fattori che possono variare i risultati di una gara e di conseguenza anche dell'intero campionato. Tra i vari esempi di fattori non considerati o che potrebbero ottenere un miglioramento possiamo trovare:
\begin{itemize}

    \item\textbf{Abilità del pilota}: questo fattore è stato considerato ma l'abilità è stata valutata attraverso una media di varie abilità possedute dal pilota. Il miglioramento potrebbe consistere nella scissione di questo valore poichè in questa competizione i layout dei vari tracciati delle varie gare sono molto diversi tra loro e questo porterebbe ad una più reale simulazione della gara. 
    \item \textbf{Usura delle gomme}: questo fattore è molto influente in F1 poichè le prestazioni di una vettura in un determinato tracciato dipendono dall'usura degli pneumatici. Questo fattore è stato considerato prendendo come riferimento per il tempo di percorrenza di ogni giro il tempo impiegato da un pilota per percorrere l'iesimo giro e quindi al netto dell'usura delle gomme. Essa, l'usura, potrebbe essere simulata per ogni vettura e ogni pilota poichè anche lo stile di guida potrebbe influenzarla e tutto ciò porterebbe ad una più reale simulazione del campionato;
    \item \textbf{Layout tracciati}: questo fattore non è stato considerato. L'introduzione di questo fattore potrebbe variare in modo significativo la simulazione poichè ogni tracciato prevede ad esempio curve dove vi è un coefficiente di difficoltà di sorpasso molto più alto rispetto a dei rettilinei.\clearpage
    \item \textbf{Strategia Team}: questo fattore potrebbe essere implementato attraverso un algoritmo di machine learning. Un Team-principal decide quali sono le strategie da adottare in gara attraverso l'analisi delle telemetrie, attraverso l'esperienze e attraverso il suo intuito. Eliminando quest'ultima componente, questo lavoro può essere effettuato da una macchina.
\end{itemize}
Infine, soffermandosi ad analizzare i tempi di esecuzione dell'applicazione, si può trarre che la durata della simulazione non sia influenzata notevolmente dal numero di gare che si decide di far disputare questo poichè si passa da un tempo stimato di circa 45 millisecondi con 24 gare ad un tempo stimato di circa 20 millisecondi con 10 gare; la durata è invece molto influenzata dalla lettura dei dati dal database. Questo accade poichè oltre al tempo di estrazione dei dati, vi è un tempo per elaborare i dati nelle apposite strutture create e il tempo stimato per questo processo è di circa 1 secondo (in media, calcolata su 10 simulazioni, è di 1096 millisecondi). Tenendo conto di queste considerazioni, si ritiene che il margine di miglioramento dei tempi di esecuzione dell'applicazione sia nella possibilità di elaborare query SQL più ottimizzate.